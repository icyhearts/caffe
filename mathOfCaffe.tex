\documentclass[openany,a4paper]{book}
\usepackage{fontspec}
\renewcommand{\baselinestretch}{1.2}

%->->->->->->->->->->->->->->->->->->->->->->->->->->->->->->
%<-<-<-<-<-<-<-<-<-<-<-<-<-<-<-<-<-<-<-<-<-<-<-<-<-<-<-<-<-<-



%->->->->->->->->->->->->->->->->->->->->->->->->->->->->->->
% hyperref 会生成书签和目录超链接
\usepackage[colorlinks,linkcolor=black]{hyperref}
%\usepackage{hyperref}
\hypersetup{
}
%<-<-<-<-<-<-<-<-<-<-<-<-<-<-<-<-<-<-<-<-<-<-<-<-<-<-<-<-<-<-
%->->->->->->->->->->->->->->->->->->->->->->->->->->->->->->
\usepackage{float}
%<-<-<-<-<-<-<-<-<-<-<-<-<-<-<-<-<-<-<-<-<-<-<-<-<-<-<-<-<-<-



%->->->->->->->->->->->->->->->->->->->->->->->->->->->->->->三线表格
\usepackage{minibox}
\usepackage{booktabs}
\usepackage{tikz}
\usepackage{float}
\usepackage{tabularx}
%<-<-<-<-<-<-<-<-<-<-<-<-<-<-<-<-<-<-<-<-<-<-
%->->->->->->->->->->->->->->->->->->->->->->->->->->->->->->三线表格
\usepackage{varwidth}
%<-<-<-<-<-<-<-<-<-<-<-<-<-<-<-<-<-<-<-<-<-<-
%->->->->->->->->->->->->->->->->->->->->->->->->->->->->->->三线表格
\usepackage{multirow}
%<-<-<-<-<-<-<-<-<-<-<-<-<-<-<-<-<-<-<-<-<-<-





%->->->->->->->->->->->->->->->->->->->->->->->->->->->->->->
\usepackage{amsmath} % math related
\usepackage{amssymb}
% \usepackage{bbold}
%<-<-<-<-<-<-<-<-<-<-<-<-<-<-<-<-<-<-<-<-<-<-<-<-<-<-<-<-<-<-

%->->->->->->->->->->->->->->->->->->->->->->->->->->->->->->
% 使用 \makeindex \printindex 命令用
\usepackage{makeidx}
\makeindex
%<-<-<-<-<-<-<-<-<-<-<-<-<-<-<-<-<-<-<-<-<-<-<-<-<-<-<-<-<-<-

%%%%%%%%%%%->->->->->->->->->->->->->->->->->->->->->->->->->->->->->->
\usepackage[numbers,square]{natbib}
\bibliographystyle{plainnat}
\setcitestyle{square}
%\setlength{\bibhang}{10cm}
\setlength{\bibindent}{10cm}
%\setlength{\bibsep}{3cm}
%<-<-<-<-<-<-<-<-<-<-<-<-<-<-<-<-<-<-<-<-<-<-<-<-<-<-<-<-<-<-
\usepackage{tocloft} %目录
\usepackage{titletoc} %目录
\usepackage{indentfirst}
\usepackage{geometry}
\geometry{left=2cm,right=2cm,top=3cm,bottom=2cm}
\usepackage{titlesec}% \titleformat
\usepackage{ctex}
%->->->->->->->->->->->->->->->->->->->->->->->->->->->->->->
\usepackage{fancyhdr}
\pagestyle{fancy}
\pagenumbering{arabic}
\renewcommand{\headrulewidth}{1pt}
\fancypagestyle{plain}
{
\fancyhf{}
\fancyfoot[OR]{\rule[-2pt]{1pt}{15pt}\hspace{5pt}page \bf\arabic{page}\hspace{11pt}\rule[3pt]{20pt}{2pt}}
\fancyfoot[EL]{\rule[3pt]{20pt}{2pt}\hspace{5pt}\bf\arabic{page}\hspace{11pt}\rule[-2pt]{1pt}{15pt}}
}

%<-<-<-<-<-<-<-<-<-<-<-<-<-<-<-<-<-<-<-<-<-<-<-<-<-<-<-<-<-<-
%->->->->->->->->->->->->->->->->->->->->->->->->->->->->->->->->->->->->->->->-> 字体相关

\setmainfont{Times New Roman}
\setCJKmainfont{SimSun}

\setCJKfamilyfont{song}{SimSun}
\setCJKfamilyfont{hei}{SimHei}
\setCJKfamilyfont{kai}{KaiTi}
\newcommand{\song}{\CJKfamily{song}}
\newcommand{\hei}{\CJKfamily{hei}}
\newcommand{\kai}{\CJKfamily{kai}}
%<-<-<-<-<-<-<-<-<-<-<-<-<-<-<-<-<-<-<-<-<-<-<-<-<-<-<-<-<-<-<-<-<-<-<-<-<-<-<-<-

%->->->->->->->->->->->->->->->->->->->->->->->->->->->->->->
\usepackage{graphicx}
\renewcommand{\figurename}{图}
\renewcommand{\thefigure}{\thechapter-\arabic{figure}} %Figure 1.1 to Figure 1-1
%\renewcommand{\thefigure}{\thesection-\arabic{figure}} %Figure 3.4-5 just like DIP book
\usepackage{caption,subcaption} % for captionsetup,subcaption用于插入多个图片时,有一个总标题,各个小图片有各自的小标题
\DeclareCaptionLabelSeparator{space}{\ \ }
\captionsetup{labelsep=space}
%<-<-<-<-<-<-<-<-<-<-<-<-<-<-<-<-<-<-<-<-<-<-<-<-<-<-<-<-<-<-

%->->->->->->->->->->->->->->->->->->->->->->->->->->->->->->
\graphicspath{{figures/}}
%<-<-<-<-<-<-<-<-<-<-<-<-<-<-<-<-<-<-<-<-<-<-<-<-<-<-<-<-<-<-
\newcommand{\SPACE}{\makebox{\begin{picture}(0,0)\put(1,0){\line(1,0){2}}\put(1,0){\line(0,1){3}}\put(3,0){\line(0,1){3}} \end{picture} } }
		\setcounter{secnumdepth}{4} % 改变标题编号的深度级别
		\setcounter{tocdepth}{4} % 改变标题编号的深度级别
\includeonly{ch1}
\begin{document}
 \titleformat{\chapter}       {\Huge}{第\,\thechapter\,章}{1em} {}
%\titleformat{command }[shape]{format } {label}              {sep }{before }[after ]
\titleformat{\subsection}       {\hei}                 {\thesubsection}        {1em}                  {}
%\titleformat{command   }[shape]{format<font,size> }   {label}                 {sep<label and title>} {before }[after ]
%\titleformat{\subsubsection}  {\Large\hei}{1,\arabic{subsubsection}}{1em}  {}

%\titlecontents{chapter}[4em]{\large}{\contentslabel{4em}}
%{}{$\cdots$\small\contentspage}
\titlecontents{chapter}% 1<section-type>
 [10pt]% 2<left>
{\vspace{2em}}% 3<above-code>
{\kai\large 第 \thecontentslabel 章\quad}% 4<numbered-entry-format>
% thecontentslabel就是那个前面第2 章的‘2’
{}% 5<numberless-entry-format>
%{{$\cdot$}\hfill\contentspage}% <filler-page-format>
%{\titlerule*[1pc]{.}\contentspage}% <filler-page-format>
{\titlerule*[1pc]{.}\contentspage}% <filler-page-format>
%{\bfseries\hfill\contentspage}% <filler-page-format>

\addcontentsline{toc}{chapter}{目录}%把目录---------1这一行加入目录
%->->->->->->->->->->->->->->->->->->->->->->->->->->->->->->开始目录
\renewcommand{\contentsname}{中文目录}
\tableofcontents
%<-<-<-<-<-<-<-<-<-<-<-<-<-<-<-<-<-<-<-<-<-<-<-<-<-<-<-<-<-<-
\newcounter{example}
%%%% fontsize
\newcommand{\chuhao}{\fontsize{42pt}{\baselineskip}\selectfont}
\newcommand{\xiaochuhao}{\fontsize{36pt}{\baselineskip}\selectfont}
\newcommand{\yihao}{\fontsize{28pt}{\baselineskip}\selectfont}
\newcommand{\erhao}{\fontsize{21pt}{\baselineskip}\selectfont}
\newcommand{\xiaoerhao}{\fontsize{18pt}{\baselineskip}\selectfont}
\newcommand{\sanhao}{\fontsize{15.75pt}{\baselineskip}\selectfont}
\newcommand{\sihao}{\fontsize{14pt}{\baselineskip}\selectfont}
\newcommand{\xiaosihao}{\fontsize{12pt}{\baselineskip}\selectfont}
\newcommand{\wuhao}{\fontsize{10.5pt}{\baselineskip}\selectfont}
\newcommand{\xiaowuhao}{\fontsize{9pt}{\baselineskip}\selectfont}
\newcommand{\liuhao}{\fontsize{7.875pt}{\baselineskip}\selectfont}
\newcommand{\qihao}{\fontsize{5.25pt}{\baselineskip}\selectfont}
%%% end of fontsize
\xiaosihao

\chapter{InnerProductLayer}
\begin{table}[!ht]
\begin{tabular}{|c|c|c|}
\hline
$N\_$ & $K\_$ & M\_\\
\hline
$N_{out}$ & $N_{in}$ & B\\
\hline
\end{tabular}
\end{table}

https://www.cnblogs.com/pinard/p/10825264.html
\begin{equation}
z=f(\mathbb{Y}), \mathbb{Y}=\mathbb{A}\mathbb{X}+\mathbb{B} \rightarrow \frac{\partial{z}}{\partial\mathbb{X}}=\mathbb{A}^T\frac{\partial z}{\partial\mathbb{Y}}
\end{equation}

\begin{equation}
z=f(\mathbb{Y}), \mathbb{Y}=\mathbb{X}\mathbb{A}+\mathbb{B} \rightarrow \frac{\partial{z}}{\partial\mathbb{X}}=\frac{\partial z}{\partial\mathbb{Y}}\mathbb{A}^T
\end{equation}

\begin{table}[!ht]
\caption{InnerProductLayer forward}
% \begin{tabularx}{\textwidth}{|X|c|c|c|c|c|}
\begin{tabular}{|l|c|c|c|c|c|}
\hline
item &-   & NoTrans & NoTrans & NoTrans & Trans\\
\hline
code &top= & bias\_multiplier\_ & *blobs\_[1]  & + bottom & *weight\\
\hline
RAM & $B\times N_{out}$ & B & $N_{out}$ & $B\times N_{in}$ &  $N_{out}\times N_{in}$\\
\hline
math & $\mathbb{Y}=$ & $\vec{1}$  & *$\vec{b}^T$ & +$\mathbb{I} $ & $*\mathbb{W}^T$ \\
\hline
math size & $B\times N_{out}$& $B\times 1$& $1\times N_{out}$ & $B\times N_{in}$ & $(N_{out}\times N_{in})^T$\\
\hline
% \end{tabularx}
\end{tabular}
\end{table}

\begin{table}[!ht]
\caption{$ \frac{ \partial e}{\partial (\mathbb{W}^T)}=\mathbb{I}^T\frac{\partial e}{\partial\mathbb{Y}} \rightarrow \frac{ \partial e}{\partial \mathbb{W}}=(\frac{\partial e}{\partial\mathbb{Y}})^T \mathbb{I} $ }
\begin{tabular}{|l|c|c|c|}
\hline
item & -   & Trans & NoTrans \\
\hline
% code &blobs\_[0]->mutable\_cpu\_diff() & top[0]-\>cpu\_diff() & bottom[0]-\>cpu\_data()\\
  code &blobs\_[0]->mutable\_cpu\_diff() & top[0]->cpu-\_diff() & bottom[0]->cpu\_data()\\
\hline
RAM size & $N_{out}\times N_{in}$ & $B\times N_{out}$ & $B\times N_{in}$\\
\hline
math& $\frac{\partial e}{\partial \mathbb{W}}= $ &$(\frac{\partial e}{\partial\mathbb{Y}})^T$ & *$\mathbb{I}$ \\
\hline
math size & $N_{out}\times N_{in}$ & $(B\times N_{out})^T$ & $B\times N_{in}$ \\
\hline
\end{tabular}
\end{table}


\begin{tabular}{|l|c|c|c|}
\hline
item & -   & Trans & NoTrans \\
\hline
% code &blobs\_[0]->mutable\_cpu\_diff() & top[0]-\>cpu\_diff() & bottom[0]-\>cpu\_data()\\
  code &blobs\_[0]->mutable\_cpu\_diff() & top[0]->cpu-\_diff() & bottom[0]->cpu\_data()\\
\hline
RAM size & $N_{out}\times N_{in}$ & $B\times N_{out}$ & $B\times N_{in}$\\
\hline
math& $\frac{\partial e}{\partial \mathbb{W}}= $ &$(\frac{\partial e}{\partial\mathbb{Y}})^T$ & *$\mathbb{I}$ \\
\hline
math size & $N_{out}\times N_{in}$ & $(B\times N_{out})^T$ & $B\times N_{in}$ \\
\hline
\end{tabular}
\end{table}




\begin{table}[!ht]
\caption{$\frac{\partial e}{\partial \vec{b}}$}
% \begin{tabularx}{\textwidth}{|X|c|c|c|c|c|}
\begin{tabular}{|l|c|c|c|}
\hline
item & -   & Trans & -(sgemv) \\
\hline
  code && & \\
\hline
RAM size & $N_{out}\times N_{in}$ & $B\times N_{out}$ & $B\times N_{in}$\\
\hline
math& $\frac{\partial e}{\partial \mathbb{W}}= $ &$(\frac{\partial e}{\partial\mathbb{Y}})^T$ & *$\mathbb{I}$ \\
\hline
math size & $N_{out}\times N_{in}$ & $(B\times N_{out})^T$ & $B\times N_{in}$ \\
\hline
\end{tabular}
\end{table}




\end{document}
%->->->->->->->->->->->->->->->->->->->->->->->->->->->->->->
%<-<-<-<-<-<-<-<-<-<-<-<-<-<-<-<-<-<-<-<-<-<-<-<-<-<-<-<-<-<-
